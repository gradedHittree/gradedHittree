\documentclass[10.5pt]{jsarticle} 
\usepackage{graphicx}
\def\vector#1{\mbox{\boldmath $#1$}} 
\setlength{\columnsep}{3zw}
\setlength{\columnseprule}{0.4pt}
\title{gradedHittree soft Docment \\
Version 2.0} 
\author{Hiroki Yoneda} 
\date{\today}
%%%%%%    TEXT START    %%%%%% 
\begin{document} 
\maketitle 

これまでのhittreeでは、一挙に行われていたmergeやreconstructなどを分解して解析することや、
それらのアルゴリズムをより柔軟に実装できるのを目標として、
"new hittree soft"を作成中である。
このDocumentでは、databaseや生成ファイルについて、説明する。

\section{Databese}
検出器に与えるデータベースは、detector\_map、detector\_profile, cal関数の3種類必要である。
\subsection{detector\_map}
ASICID, ASICCHとDETID, DETCHとの対応付をするデータである。
"detector\_map"というTTree形式で与えられる。
以下のようなBranchを持っている。
\begin{table}[htb]
  \begin{tabular}{|c|c|p{12cm}|} \hline
  Branch名 & 型 & 説明 \\ \hline
  asicid & Int\_t & \\
  asicch & Int\_t & 使用していないchであっても記述する必要あり\\
  remapch & Int\_t & 検出器全体でのchに対する通し番号。使用していないchは、"-1"を記入。\\
  detid & Int\_t  & 検出器のID。Siを0-9に、CdTeを10以上にすることを推奨。\\
  detch & Int\_t &  検出器内でのCh。 隣接Chは、detchも隣接する。使用していないchは、"-1"を記入。\\ \hline
  \end{tabular}
\end{table}
また、asicは、64ch持っていることを前提している。

\subsection{detector\_profile}
検出器は、xy面に平行であり、各ストリップは、x軸、または、y軸に平行であることを前提としている。
検出器が3次元的に複雑に配置される場合は、hittree\_lv3から変換する必要あり。
\begin{table}[htb]
  \begin{tabular}{|c|c|p{10cm}|} \hline
  Branch名 & 型 & 説明 \\ \hline
  detid & Int\_t  & 検出器のID。Siを0-9に、CdTeを10以上にすることを推奨。\\
  detch & Int\_t &  使用していないchは、ここでは記入する必要なし。\\
  detector\_material & Int\_t & 0: Si、1:CdTe\\
  detector\_HV & Int\_t & 0: Ground 1: HVside\\
  pos\_x  & Double\_t & 位置情報を持たない場合は、適当な値(0など)を入れる。\\
   pos\_y  & Double\_t & 位置情報を持たない場合は、適当な値(0など)を入れる。\\
   pos\_z  & Double\_t & \\
   delta\_x  & Double\_t & 位置情報を持たない場合は、"-1"を入れる。負値から、ストリップ方向を認識する。\\
   delta\_y  & Double\_t & 位置情報を持たない場合は、"-1"を入れる。負値から、ストリップ方向を認識する。\\
   delta\_z  & Double\_t & \\
   badch & Int\_t & 0: Good Channel 1: Bad Channel \\
   ethre & Double\_t  (keV) & チャンネルごとに設定したエネルギースレッショルド。lv2ファイル生成時に使うことを想定。 \\
   \hline
  \end{tabular}
\end{table}

\subsection{cal関数}
"calfunc\_DETID\_DETCH"という名前のTSpline3を持っているroot fileで与えられる。

\section{生成ファイル}
データの生成は、
eventtree $\to$
hitttree\_lv1 $\to$
hitttree\_lv2 $\to$
hitttree\_lv3
の3段階に分かれて行う。

hitttree\_lv1 は、データベースの適用を行うのみとする。
hitttree\_lv2 は、ストリップのマージを基本的には行う。
hitttree\_lv3 は、マージされたストリップシグナルを元に、ヒットを再構成する。
各段階で、新しいBranchが元のtreeに追加されていく形を採用する。
したがって、hittree\_lv3は、eventtree, hitttree\_lv1, hitttree\_lv2の情報も保持している。

\subsection{hitttree\_lv1}
エネルギースレッショルドは適用せず、データベースを当てるのみ。
各ブランチは、データベースの値をコピーした値になっている。
1番目以外は、全て可変長配列。
\begin{table}[htb]
\begin{center}
  \begin{tabular}{|c|c|c|} \hline
  型 & Branch名  & 説明 \\ \hline
        Int\_t & ndetector\_lv1& \\
        Int\_t & detid\_lv1[ndetector\_lv1]& \\
        Int\_t & material\_lv1[ndetector\_lv1]& \\
        Int\_t & nsignal\_x\_lv1[ndetector\_lv1]& \\
        Int\_t & nsignal\_y\_lv1[ndetector\_lv1]& \\
        Int\_t & detch\_x\_lv1[ndetector\_lv1][MAX\_N\_SIGNAL]& \\
        Int\_t & detch\_y\_lv1[ndetector\_lv1][MAX\_N\_SIGNAL]& \\
        Double\_t & epi\_x\_lv1[ndetector\_lv1][MAX\_N\_SIGNAL]& \\
        Double\_t & epi\_y\_lv1[ndetector\_lv1][MAX\_N\_SIGNAL]& \\
        Double\_t & ethre\_x\_lv1[ndetector\_lv1][MAX\_N\_SIGNAL]& \\
        Double\_t & ethre\_y\_lv1[ndetector\_lv1][MAX\_N\_SIGNAL]& \\
        Double\_t & pos\_x\_lv1[ndetector\_lv1][MAX\_N\_SIGNAL]& \\
        Double\_t & delta\_x\_lv1[ndetector\_lv1][MAX\_N\_SIGNAL]& \\
        Double\_t & pos\_y\_lv1[ndetector\_lv1][MAX\_N\_SIGNAL]& \\
        Double\_t & delta\_y\_lv1[ndetector\_lv1][MAX\_N\_SIGNAL]& \\
        Double\_t & pos\_z\_lv1[ndetector\_lv1]& \\
        Double\_t & delta\_z\_lv1[ndetector\_lv1]& \\ 
   \hline
  \end{tabular}
    \caption{hittree\_lv1}
    \end{center}
\end{table}

\subsection{hitttree\_lv2}
マージアルゴリズムを適用した後のシグナル情報。
posやdeltaは、マージアルゴリズム内で計算する必要あり。
\begin{table}[htb]
\begin{center}
  \begin{tabular}{|c|c|c|} \hline
型 & Branch名 & 説明 \\ \hline
Int\_t & ndetector\_lv2& \\
Int\_t & detid\_lv2[ndetector\_lv2]& \\
Int\_t & material\_lv2[ndetector\_lv2]& \\
Int\_t &nsignal\_x\_lv2[ndetector\_lv2]& \\
Int\_t &nsignal\_y\_lv2[ndetector\_lv2]& \\
Int\_t &n\_merged\_strips\_x\_lv2[ndetector\_lv2][MAX\_N\_SIGNAL]& \\
Int\_t &n\_merged\_strips\_y\_lv2[ndetector\_lv2][MAX\_N\_SIGNAL]& \\
Int\_t &detch\_array\_x\_lv2[ndetector\_lv2][MAX\_N\_SIGNAL][MAX\_N\_SIGNAL\_2]& \\
Int\_t &detch\_array\_y\_lv2[ndetector\_lv2][MAX\_N\_SIGNAL][MAX\_N\_SIGNAL\_2]& \\
Double\_t& epi\_array\_x\_lv2[ndetector\_lv2][MAX\_N\_SIGNAL][MAX\_N\_SIGNAL\_2]& \\
Double\_t& epi\_array\_y\_lv2[ndetector\_lv2][MAX\_N\_SIGNAL][MAX\_N\_SIGNAL\_2]& \\
Double\_t& epi\_x\_lv2[ndetector\_lv2][MAX\_N\_SIGNAL]& \\
Double\_t& epi\_y\_lv2[ndetector\_lv2][MAX\_N\_SIGNAL]& \\
Double\_t& pos\_x\_lv2[ndetector\_lv2][MAX\_N\_SIGNAL]& \\
Double\_t& delta\_x\_lv2[ndetector\_lv2][MAX\_N\_SIGNAL]& \\
Double\_t& pos\_y\_lv2[ndetector\_lv2][MAX\_N\_SIGNAL]& \\
Double\_t& delta\_y\_lv2[ndetector\_lv2][MAX\_N\_SIGNAL]& \\
Double\_t& pos\_z\_lv2[ndetector\_lv2]& \\
Double\_t& delta\_z\_lv2[ndetector\_lv2]& \\
 \hline
  \end{tabular}
  \caption{hittree\_lv2}
   \end{center}
\end{table}

\subsection{hitttree\_lv3}
両面情報からヒット情報に変換した後のデータ。
posやdeltaは、再構成アルゴリズム内で計算する必要あり。

\begin{table}[htb]
\begin{center}
\begin{tabular}{|c|c|c|} \hline
    型 & Branch名 & 説明 \\ \hline
Int\_t & ndetector\_lv3& \\
Int\_t & detid\_lv3[ndetector\_lv3]& \\
Int\_t & material\_lv3[ndetector\_lv3]& \\
Int\_t & nsignal\_x\_lv3[ndetector\_lv3]& \\
Int\_t & nsignal\_y\_lv3[ndetector\_lv3]& \\
Int\_t & nhit\_lv3[ndetector\_lv3]& \\
Int\_t & n\_merged\_strips\_x\_lv3[ndetector\_lv3][MAX\_N\_HIT]& \\
Int\_t & n\_merged\_strips\_y\_lv3[ndetector\_lv3][MAX\_N\_HIT]& \\
Double\_t & epi\_x\_lv3[ndetector\_lv3][MAX\_N\_HIT]& \\
Double\_t & epi\_y\_lv3[ndetector\_lv3][MAX\_N\_HIT]& \\
Double\_t & epi\_reconstructed\_lv3[ndetector\_lv3][MAX\_N\_HIT]& \\
Double\_t & pos\_x\_lv3[ndetector\_lv3][MAX\_N\_HIT]& \\
Double\_t & delta\_x\_lv3[ndetector\_lv3][MAX\_N\_HIT]& \\
Double\_t & pos\_y\_lv3[ndetector\_lv3][MAX\_N\_HIT]& \\
Double\_t & delta\_y\_lv3[ndetector\_lv3][MAX\_N\_HIT]& \\
Double\_t & pos\_z\_lv3[ndetector\_lv3][MAX\_N\_HIT]& \\
Double\_t & delta\_z\_lv3[ndetector\_lv3][MAX\_N\_HIT]& \\
\hline
  \end{tabular}
      \caption{hittree\_lv3}
 \end{center}
\end{table}

\section{その他}
配列や、ASICS数の最大値などを{\tt src/ConstantPar.hh}で設定している。
実験システムが大きい場合は、ここを見直して、十分な値になっているか確認したほうがよい。

\end{document}
